% define a classe de documento, no caso um artigo em fonte de 12pt
\documentclass[12pt]{article}

% Carrega 
\input{packages_latex}


\addbibresource{InforSoc.bib}

\begin{document} 

\title{Sistema de Gerenciamento de Consultas Veterinárias}
	
\author{Ana Luiza Campos Souza \\ Célio Júnio de Freitas Eduardo \\
Giulia Moura Ferreira \\ Letícia}
		
\newcommand{\address}{{
			\begin{center}
				\footnotesize
				Universidade de Brasília - Instituto de Ciências Exatas\\  Departamento de Ciência da Computação - CIC0134 - Banco de Dados \\
				Professora Maristela Holanda \\ Prédio CIC/EST - Campus Universitário Darcy Ribeiro \\Asa Norte 70919-970 Brasília, DF\\
				\href{seu-email@aluno.unb.br}{211055441@aluno.unb.br}
			\end{center}
}}
	
\maketitle
\address
	
\begin{abstract} 
		Após concluir o projeto, escreva aqui o resumo usando um único parágrafo de texto, com até 500 palavras. \\
\textbf{Palavras-chave:} \textit{Informática e Sociedade; UnB; Ensaio Científico; Dilemas; Ética.}
\end{abstract}

\section{Introdução}

Espaço destinado a introdução.

\section{Metodologia}

Descreva a metodologia de \cite{jones_doing_2016}, isto é, os passos que você deve seguir para realizar o trabalho, conforme as orientações do Plano de Ensino.

\section{Problemas e Dilemas Ético-Morais}

Defina o que são problemas e dilemas ético-morais no campo da informática e sociedade, conforme as orientações do Plano de Ensino.

Veja o esquema geral de um dilema na Figura \ref{fig:dilema}, apresentado de forma mais detalhada no Plano de Ensino.

\begin{figure}
    \centering
    \includegraphics[width=\textwidth]{images/DilemaEticoInformatica.png}
    \caption{Modelo Geral de um Dilema Ético-Moral em Informática. Fonte: \cite{fernandes_texto_2024-1}}
    \label{fig:dilema}
\end{figure}

\section{Análise Tecnológica} 

Escreva um texto que desconstrua e analise a tecnologia (artefato informático) relacionada com o dilema proposto, sob as várias perspectivas (análise \textit{multistakeholder}), conforme as orientações do Plano de Ensino.
	
\section{Análise Ética}
	
Declare um dilema concreto e faça uma análise valorativa dele, conforme as orientações do Plano de Ensino.
 
\section{Análise Jurídica, Legal e Normativa}
	
Pesquise, cite e descreva leis, normas e regulamentos que auxiliam na análise do dilema, sendo um dos exemplos a LGPD \cite{brasil_lei_2018}, 
conforme as orientações do Plano de Ensino.

\section{Análise Profissional e Empresarial}
	
Analise as profissões e empresas envolvidas no dilema, e o que dizem seus códigos de conduta acerca dos elementos do dilema, 
conforme as orientações do Plano de Ensino.

\section{Proposições e Encaminhamentos}
	
Apresentar encaminhamentos para a solução do dilema analisado, ponderando valores e princípios sociais,
conforme as orientações do Plano de Ensino.

\section{Conclusões sobre a disciplina}
	
Nas conclusões do ensaio você deve refletir e relatar sobre o que ocorreu no semestre, conforme orienta o Plano de Ensino.

\label{referencias}
\printbibliography
\end{document}